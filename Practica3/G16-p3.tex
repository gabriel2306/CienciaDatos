\documentclass [a4paper] {article}

\usepackage[spanish]{babel} 
\usepackage[utf8]{inputenc} 
\usepackage{multirow} 
\usepackage{float} 

\title{R-PL3}
\author{Gabriel López, Sergio Sanz, Álvaro Zamorano}

\usepackage{Sweave}
\begin{document}
\Sconcordance{concordance:G16-p3.tex:G16-p3.Rnw:%
1 10 1 1 0 12 1 1 2 6 0 1 1 6 0 1 2 2 1 1 2 1 0 1 1 3 0 1 2 35 1 1 2 4 %
0 1 2 3 1 1 2 4 0 1 2 5 1 1 2 1 0 1 1 15 0 1 2 3 1 1 2 6 0 1 1 3 0 1 2 %
1 1 1 2 1 0 1 1 12 0 1 2 2 1 1 2 4 0 1 2 3 1 1 2 1 0 1 1 13 0 1 2 24 1 %
1 2 1 0 1 1 3 0 1 2 5 1 1 2 1 0 1 1 11 0 1 2 6 1 1 2 1 0 2 1 3 0 1 2 2 %
1 1 2 1 0 1 1 3 0 1 2 33 1 1 2 1 0 1 1 3 0 1 2 3 1 1 2 1 0 1 1 17 0 1 2 %
2 1 1 2 4 0 1 2 3 1 1 2 1 0 1 1 15 0 1 2 9 1 1 2 1 0 1 1 3 0 1 2 2 1 1 %
2 11 0 1 1 10 0 1 1 10 0 1 1 11 0 1 2 6 1 1 2 1 0 1 1 19 0 1 2 2 1 1 2 %
4 0 1 2 2 1}


\maketitle

\graphicspath{ {./tmp/} }

\section{Ejercicio realizado en clase.}
Para obtener la \textbf{función de clasificación} mediante el algoritmo construcción
de \textbf{árboles de decisión de Hunt} es necesario usar los paquetes \texttt{rpart} y
\texttt{tree}. Estos paquetes hay que descargarlos desde la página de CRAN y para instalarlos
hay que ejecutar el siguiente código:

\begin{Schunk}
\begin{Sinput}
> install.packages("./Paquetes/rpart_4.1-15.zip")
> install.packages("./Paquetes/tree_1.0-40.zip")
\end{Sinput}
\end{Schunk}

\bigskip
De esta forma, los paquete únicamente estarán instalados. Para poder usarlos es necesario cargarlos:
\begin{Schunk}
\begin{Sinput}
> library(rpart)
> library(tree)
\end{Sinput}
\end{Schunk}

\begin{itemize}
\item Con rpart obtendremos las particiones recursivas para la clasificación y los árboles de decisión.
\item Con tree, los árboles de clasificación y regresión.
\end{itemize}

%%%%%%%%%%%%%%%%%%%%%%%%%%%%%%%%%%%%%%%%%%%%%%%%%%% 1.1 %%%%%%%%%%%%%%%%%%%%%%%%%%%%%%%%%%%%%%%%%%%%%%%%%%%

\subsection{Función de clasificación.}
Los datos a usar en este primer ejercicio se componen de 9 calificaciones de estudiantes compuestas por
Teoría, Laboratorio, Prácticas y Calificación Global.

\bigskip
Para introducir estos datos en el algoritmo a usar es necesario tener un fichero \texttt{.txt} con el
siguiente aspecto.
\begin{table}[H]
\begin{center}
\begin{tabular}{|c|c|c|c|c|}
\hline
Suceso & Teoría & Lab & Prac & Calif\\
\hline \hline
s1 & A & A & B & Ap \\ \hline
s2 & A & B & D & Ss \\ \hline
s3 & D & D & C & Ss \\ \hline
s4 & D & D & A & Ss \\ \hline
s5 & B & C & B & Ss \\ \hline
s6 & C & B & B & Ap \\ \hline
s7 & B & B & A & Ap \\ \hline
s8 & C & D & C & Ss \\ \hline
s9 & B & A & C & Ss \\ \hline
\end{tabular}
\end{center}
\end{table}

\bigskip
Procedemos a leer dicho fichero .txt mediante el uso de la función \textit{read.table.}
\begin{Schunk}
\begin{Sinput}
> calificaciones<-read.table("./Datos/Calificaciones.txt")
\end{Sinput}
\end{Schunk}

\bigskip
Para asegurarnos de que todo irá bien a la hora de realizar la clasificación, convertimos los datos
leídos en un \textbf{dataframe.}
\begin{Schunk}
\begin{Sinput}
> muestra<-data.frame(calificaciones)
\end{Sinput}
\end{Schunk}

\bigskip
Nuestros datos ya se encuentran preparados para aplicarles la función \textbf{rpart.} Es importante destacar
el uso de minsplit ya que disponemos de una muestra con un número muy reducido de datos. Por otra parte, la función
rpart usa como medida de impureza por defecto Gini.
\begin{Schunk}
\begin{Sinput}
> clasificacion<-rpart(Calif~.,data=muestra,method="class",minsplit=1)
> clasificacion
\end{Sinput}
\begin{Soutput}
n= 9 

node), split, n, loss, yval, (yprob)
      * denotes terminal node

1) root 9 3 Ss (0.3333333 0.6666667)  
  2) Lab=A,B 5 2 Ap (0.6000000 0.4000000)  
    4) Prac=A,B 3 0 Ap (1.0000000 0.0000000) *
    5) Prac=C,D 2 0 Ss (0.0000000 1.0000000) *
  3) Lab=C,D 4 0 Ss (0.0000000 1.0000000) *
\end{Soutput}
\end{Schunk}

\bigskip
Para mostrar el árbol de clasificación hacemos uso de una función que hemos definido, pero para poder usarla 
en primer lugar es necesario instalar el paquete \textbf{rpart.plot.}
\begin{Schunk}
\begin{Sinput}
> install.packages("./Paquetes/rpart.plot_3.0.8.zip")
> library(rpart.plot)
\end{Sinput}
\end{Schunk}

Dicha función es:
\begin{Schunk}
\begin{Sinput}
> source("Funciones/plotTree.R")
> plotTree
\end{Sinput}
\begin{Soutput}
function(tree, ruta) {

    png(paste("./tmp/",ruta,sep=""))

    rpart.plot(tree, box.palette="RdBu", shadow.col="gray", nn=TRUE)

    dev.off()
}
\end{Soutput}
\end{Schunk}

\bigskip
Procedemos a su ejecución.
\begin{Schunk}
\begin{Sinput}
> plotTree(clasificacion, "classTree.png")
\end{Sinput}
\end{Schunk}
\includegraphics[width=\textwidth]{classTree}

\bigskip
Por último, se aplica la función \textbf{tree} a nuestros datos, para que esta use la medida de impureza \texttt{Gini} lo indicamos
en el parámetro split.
\begin{Schunk}
\begin{Sinput}
> clasificaciontree<-tree(Calif~.,data=muestra,mincut=1,minsize=2,split="gini")
> clasificaciontree
\end{Sinput}
\begin{Soutput}
node), split, n, deviance, yval, (yprob)
      * denotes terminal node

1) root 9 11.46 Ss ( 0.3333 0.6667 )  
  2) Lab: A,B 5  6.73 Ap ( 0.6000 0.4000 )  
    4) Prac: A,B 3  0.00 Ap ( 1.0000 0.0000 ) *
    5) Prac: C,D 2  0.00 Ss ( 0.0000 1.0000 ) *
  3) Lab: C,D 4  0.00 Ss ( 0.0000 1.0000 ) *
\end{Soutput}
\end{Schunk}

\begin{Schunk}
\begin{Sinput}
> source("./Funciones/plotT.R")
> plotT(clasificaciontree, "classT.png")
\end{Sinput}
\end{Schunk}
\includegraphics[width=\textwidth]{classT}

%%%%%%%%%%%%%%%%%%%%%%%%%%%%%%%%%%%%%%%%%%%%%%%%%%% 1.2 %%%%%%%%%%%%%%%%%%%%%%%%%%%%%%%%%%%%%%%%%%%%%%%%%%%

\subsection{Análisis de regresión lineal.}
En este caso trabajaremos con datos de planetas, en concreto su Radio y su Diámetro. Los planetas de los que se tienen
los datos son: Mercurio, Venus, Tierra y Marte, y el .txt del que se leen dichos datos tiene el aspecto que
sigue.
\begin{table}[H]
\begin{center}
\begin{tabular}{|c|c|c|}
\hline
Planeta & Radio & Diámetro\\
\hline \hline
Mercurio & 2.4 & 5.4 \\ \hline
Venus & 6.1 & 5.2 \\ \hline
Tierra & 6.4 & 5.5 \\ \hline
Marte & 3.4 & 3.9 \\ \hline
\end{tabular}
\end{center}
\end{table}

\bigskip
Al igual que anteriormente, es necesario leer dicho fichero y pasarlo a dataframe.
\begin{Schunk}
\begin{Sinput}
> planetas<-read.table("./Datos/Planetas.txt")
> muestraP<-data.frame(planetas)
\end{Sinput}
\end{Schunk}

\bigskip
El análisis de regresión se hace mediante el uso de la función \textbf{lm} contenida en el paquete stats. Cabe destacar que el primero
de sus argumentos es de tipo \textit{fórmula} donde una expresión de la forma y \textasciitilde{} model se interpreta como una especificación de que 
la respuesta \texttt{y} está modelada por un predictor lineal especificado simbólicamente por model, es decir, en nuestro caso model=x por lo que
su ejecución queda como:
\begin{Schunk}
\begin{Sinput}
> regresionP<-lm(D~R,data=muestraP)
> regresionP
\end{Sinput}
\begin{Soutput}
Call:
lm(formula = D ~ R, data = muestraP)

Coefficients:
(Intercept)            R  
     4.3624       0.1394  
\end{Soutput}
\end{Schunk}

\bigskip
De acuerdo a la ecuación de una recta \texttt{y=a+b*x}, el primero de los coeficientes es el término independiente (a), y el segundo
de ellos la b.

\bigskip
Para mostrar el gráfico de dispersión y la recta de ajuste es necesario hacer uso de varias librerías.
\begin{Schunk}
\begin{Sinput}
> library(foreign)
> library(ggplot2)
> library(psych)
\end{Sinput}
\end{Schunk}

\bigskip
Estas librerías se usan en funciones externas usadas para representar los gráficos requeridos.
\begin{Schunk}
\begin{Sinput}
> source("Funciones/plotDisp.R")
> plotDisp(planetas,regresionP,"Radio","Diametro","regPlanetas.png")
\end{Sinput}
\end{Schunk}
\includegraphics[width=\textwidth]{regPlanetas}

%%%%%%%%%%%%%%%%%%%%%%%%%%%%%%%%%%%%%%%%%%%%%%%%%%% 2.1 %%%%%%%%%%%%%%%%%%%%%%%%%%%%%%%%%%%%%%%%%%%%%%%%%%%

\section{Segunda parte}
\subsection{Función de clasificación.}
En este ejercicio usaremos datos correspondientes a ventas de coches, en concreto son 10 muestras compuestas por: TipoCarnet, 
NúmeroRuedas, NúmeroPasajeros y TipoVehículo.

\bigskip
Para introducir estos datos en el algoritmo a usar es necesario tener un fichero \texttt{.txt} con el
siguiente aspecto.
\begin{table}[H]
\begin{center}
\begin{tabular}{|c|c|c|c|c|}
\hline
Vehículo & TipoCarnet & NúmeroRuedas & NúmeroPasajeros & TipoVehículo\\
\hline \hline
v1 & B & 4 & 5 & Coche \\ \hline
v2 & A & 2 & 2 & Moto \\ \hline
v3 & N & 2 & 1 & Bicicleta \\ \hline
v4 & B & 6 & 4 & Camión \\ \hline
v5 & B & 4 & 6 & Coche \\ \hline
v6 & B & 4 & 4 & Coche \\ \hline
v7 & N & 2 & 2 & Bicicleta \\ \hline
v8 & B & 2 & 1 & Moto \\ \hline
v9 & B & 6 & 2 & Camión \\ \hline
v10 & N & 2 & 1 & Bicicleta \\ \hline
\end{tabular}
\end{center}
\end{table}

\bigskip
Procedemos a leer dicho fichero como anteriormente y pasarlo a dataframe.
\begin{Schunk}
\begin{Sinput}
> vehiculos<-read.table("./Datos/Vehiculos.txt")
> muestraV<-data.frame(vehiculos)
\end{Sinput}
\end{Schunk}

\bigskip
Nuestros datos ya se encuentran preparados para aplicarles la función \textbf{rpart.} La clasificación a obtener será el tipo de
vehículo al que pertenece cada uno de ellos.
\begin{Schunk}
\begin{Sinput}
> clasV<-rpart(TV~.,data=muestraV,method="class",minsplit=1)
> clasV
\end{Sinput}
\begin{Soutput}
n= 10 

node), split, n, loss, yval, (yprob)
      * denotes terminal node

 1) root 10 7 Bicicleta (0.3000000 0.2000000 0.3000000 0.2000000)  
   2) TC=N 3 0 Bicicleta (1.0000000 0.0000000 0.0000000 0.0000000) *
   3) TC=A,B 7 4 Coche (0.0000000 0.2857143 0.4285714 0.2857143)  
     6) NR>=3 5 2 Coche (0.0000000 0.4000000 0.6000000 0.0000000)  
      12) NR>=5 2 0 Camion (0.0000000 1.0000000 0.0000000 0.0000000) *
      13) NR< 5 3 0 Coche (0.0000000 0.0000000 1.0000000 0.0000000) *
     7) NR< 3 2 0 Moto (0.0000000 0.0000000 0.0000000 1.0000000) *
\end{Soutput}
\end{Schunk}

\bigskip
Procedemos a mostrar el árbol de clasificación.
\begin{Schunk}
\begin{Sinput}
> plotTree(clasV, "classTreeV.png")
\end{Sinput}
\end{Schunk}
\includegraphics[width=\textwidth]{classTreeV}

\bigskip
Por último, se aplica la función \textbf{tree} a nuestros datos.
\begin{Schunk}
\begin{Sinput}
> classTV<-tree(TV~.,data=muestraV,mincut=1,minsize=2,split="gini")
> classTV
\end{Sinput}
\begin{Soutput}
node), split, n, deviance, yval, (yprob)
      * denotes terminal node

 1) root 10 27.320 Bicicleta ( 0.3000 0.2000 0.3000 0.2000 )  
   2) NP < 1.5 3  3.819 Bicicleta ( 0.6667 0.0000 0.0000 0.3333 )  
     4) TC: B 1  0.000 Moto ( 0.0000 0.0000 0.0000 1.0000 ) *
     5) TC: N 2  0.000 Bicicleta ( 1.0000 0.0000 0.0000 0.0000 ) *
   3) NP > 1.5 7 17.880 Coche ( 0.1429 0.2857 0.4286 0.1429 )  
     6) NP < 3 3  6.592 Bicicleta ( 0.3333 0.3333 0.0000 0.3333 )  
      12) TC: A 1  0.000 Moto ( 0.0000 0.0000 0.0000 1.0000 ) *
      13) TC: B,N 2  2.773 Bicicleta ( 0.5000 0.5000 0.0000 0.0000 )  
        26) TC: B 1  0.000 Camion ( 0.0000 1.0000 0.0000 0.0000 ) *
        27) TC: N 1  0.000 Bicicleta ( 1.0000 0.0000 0.0000 0.0000 ) *
     7) NP > 3 4  4.499 Coche ( 0.0000 0.2500 0.7500 0.0000 )  
      14) NR < 5 3  0.000 Coche ( 0.0000 0.0000 1.0000 0.0000 ) *
      15) NR > 5 1  0.000 Camion ( 0.0000 1.0000 0.0000 0.0000 ) *
\end{Soutput}
\end{Schunk}

\begin{Schunk}
\begin{Sinput}
> plotT(classTV, "classTV.png")
\end{Sinput}
\end{Schunk}
\includegraphics[width=\textwidth]{classTV}

%%%%%%%%%%%%%%%%%%%%%%%%%%%%%%%%%%%%%%%%%%%%%%%%%%% 2.2 %%%%%%%%%%%%%%%%%%%%%%%%%%%%%%%%%%%%%%%%%%%%%%%%%%%
\subsection{Análisis de regresión lineal.}
En este caso tenemos que hacer un análisis de regresión lineal para 4 muestras distintas compuestas
por pares de datos.

\bigskip
Como en ocasiones anteriores, procedemos a leer dichas muestras y pasarlas a dataframe.
\begin{Schunk}
\begin{Sinput}
> pares<-read.table("./Datos/Pares.txt")
> muestraPS<-data.frame(pares)
\end{Sinput}
\end{Schunk}

\bigskip
El análisis de regresión se hace mediante el uso de la función \textbf{lm.} En este caso será neceario hacer cuatro análisis diferentes.
\begin{Schunk}
\begin{Sinput}
> (r1<-lm(V2~V1,data=muestraPS))
\end{Sinput}
\begin{Soutput}
Call:
lm(formula = V2 ~ V1, data = muestraPS)

Coefficients:
(Intercept)           V1  
     3.0001       0.5001  
\end{Soutput}
\begin{Sinput}
> (r2<-lm(V4~V3,data=muestraPS))
\end{Sinput}
\begin{Soutput}
Call:
lm(formula = V4 ~ V3, data = muestraPS)

Coefficients:
(Intercept)           V3  
      3.001        0.500  
\end{Soutput}
\begin{Sinput}
> (r3<-lm(V6~V5,data=muestraPS))
\end{Sinput}
\begin{Soutput}
Call:
lm(formula = V6 ~ V5, data = muestraPS)

Coefficients:
(Intercept)           V5  
     3.0025       0.4997  
\end{Soutput}
\begin{Sinput}
> (r4<-lm(V8~V7,data=muestraPS))
\end{Sinput}
\begin{Soutput}
Call:
lm(formula = V8 ~ V7, data = muestraPS)

Coefficients:
(Intercept)           V7  
     3.0017       0.4999  
\end{Soutput}
\end{Schunk}

\bigskip
De acuerdo a la ecuación de una recta \texttt{y=a+b*x}, el primero de los coeficientes es el término independiente (a), y el segundo
de ellos la b.

\bigskip
Mostraremos estos análisis en una misma figura mediante el uso de:
\begin{Schunk}
\begin{Sinput}
> source("Funciones/plotDisp2.R")
> plotDisp2
\end{Sinput}
\begin{Soutput}
function(data, r1, r2, r3, r4, ruta) {

    png(paste("./tmp/",ruta,sep=""))

    par(mfrow = c(2, 2))

    plot(data[,1], data[,2], xlab="x", ylab="y", main="Muestra 1")
    abline(r1, col = "red")

    plot(data[,3], data[,4], xlab="x", ylab="y", main="Muestra 2")
    abline(r2, col = "blue")

    plot(data[,5], data[,6], xlab="x", ylab="y", main="Muestra 3")
    abline(r3, col = "green")

    plot(data[,7], data[,8], xlab="x", ylab="y", main="Muestra 4")
    abline(r4, col = "yellow")

    dev.off()
}
\end{Soutput}
\end{Schunk}

\bigskip
Procedemos a su ejecución.
\begin{Schunk}
\begin{Sinput}
> plotDisp2(pares, r1, r2, r3, r4, "rPares.png")
\end{Sinput}
\end{Schunk}
\includegraphics[width=\textwidth]{rPares}

%%%%%%%%%%%%%%%%%%%%%%%%%%%%%%%%%%%%%%%%%%%%%%%%%%% 2.3 %%%%%%%%%%%%%%%%%%%%%%%%%%%%%%%%%%%%%%%%%%%%%%%%%%%
\subsection{Desarrollo por parte del alumno.}

\bigskip
Para la parte de \textbf{clasificación} usaremos datos de \textit{Setas}, las cuales clasificaremos en función de si son venenosas (p) o 
comestibles (e). Procedemos a leer dicho fichero como en prácticas anteriores haciendo uso de la librería readr.
\begin{Schunk}
\begin{Sinput}
> library("readr")
> setas<-read.csv("./Datos/mushrooms.csv")
\end{Sinput}
\end{Schunk}

\bigskip
Los atributos que encontramos en este documento son:
\begin{itemize}
\item classes: edible=e, poisonous=p
\item cap-shape: bell=b,conical=c,convex=x,flat=f, knobbed=k,sunken=s
\item cap-surface: fibrous=f,grooves=g,scaly=y,smooth=s
\item cap-color: brown=n,buff=b,cinnamon=c,gray=g,green=r,pink=p,purple=u,red=e,white=w,yellow=y
\item bruises: bruises=t,no=f
\item odor: almond=a,anise=l,creosote=c,fishy=y,foul=f,musty=m,none=n,pungent=p,spicy=s
\item gill-attachment: attached=a,descending=d,free=f,notched=n
\item gill-spacing: close=c,crowded=w,distant=d
\item gill-size: broad=b,narrow=n
\item gill-color: black=k,brown=n,buff=b,chocolate=h,gray=g, green=r,orange=o,pink=p,purple=u,red=e,white=w,yellow=y
\item stalk-shape: enlarging=e,tapering=t
\item stalk-root: bulbous=b,club=c,cup=u,equal=e,rhizomorphs=z,rooted=r,missing=?
\item stalk-surface-above-ring: fibrous=f,scaly=y,silky=k,smooth=s
\item stalk-surface-below-ring: fibrous=f,scaly=y,silky=k,smooth=s
\item stalk-color-above-ring: brown=n,buff=b,cinnamon=c,gray=g,orange=o,pink=p,red=e,white=w,yellow=y
\item stalk-color-below-ring: brown=n,buff=b,cinnamon=c,gray=g,orange=o,pink=p,red=e,white=w,yellow=y
\item veil-type: partial=p,universal=u
\item veil-color: brown=n,orange=o,white=w,yellow=y
\item ring-number: none=n,one=o,two=t
\item ring-type: cobwebby=c,evanescent=e,flaring=f,large=l,none=n,pendant=p,sheathing=s,zone=z
\item spore-print-color: black=k,brown=n,buff=b,chocolate=h,green=r,orange=o,purple=u,white=w,yellow=y
\item population: abundant=a,clustered=c,numerous=n,scattered=s,several=v,solitary=y
\item habitat: grasses=g,leaves=l,meadows=m,paths=p,urban=u,waste=w,woods=d
\end{itemize}

\bigskip
\begin{Schunk}
\begin{Sinput}
> clasM<-rpart(class~.,data=setas,method="class",parms=list(split=c("information","gini")))
> clasM
\end{Sinput}
\begin{Soutput}
n= 8124 

node), split, n, loss, yval, (yprob)
      * denotes terminal node

1) root 8124 3916 e (0.51797144 0.48202856)  
  2) odor=a,l,n 4328  120 e (0.97227357 0.02772643)  
    4) spore.print.color=b,h,k,n,o,u,w,y 4256   48 e (0.98872180 0.01127820) *
    5) spore.print.color=r 72    0 p (0.00000000 1.00000000) *
  3) odor=c,f,m,p,s,y 3796    0 p (0.00000000 1.00000000) *
\end{Soutput}
\end{Schunk}

\bigskip
Procedemos a mostrar el árbol de clasificación.
\begin{Schunk}
\begin{Sinput}
> plotTree(clasM, "classTreeM.png")
\end{Sinput}
\end{Schunk}
\includegraphics[width=\textwidth]{classTreeM}

\bigskip
Función de clasificación obtenida:
\begin{itemize}
\item 
\end{itemize}

\end{document}
