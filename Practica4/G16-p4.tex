\documentclass [a4paper] {article}

\usepackage[spanish]{babel} 
\usepackage[utf8]{inputenc} 
\usepackage{multirow} 
\usepackage{float} 

\title{R-PL4}
\author{Gabriel López, Sergio Sanz, Álvaro Zamorano}

\usepackage{Sweave}
\begin{document}
\Sconcordance{concordance:G16-p4.tex:G16-p4.Rnw:%
1 10 1 1 0 32 1 1 2 1 0 1 1 14 0 1 2 7 1 1 2 1 0 1 1 8 0 1 2 3 1 1 2 10 %
0 1 2 3 1 1 2 24 0 1 2 8 1 1 2 15 0 1 2 2 1 1 2 1 0 1 1 3 0 1 2 3 1 1 2 %
10 0 1 1 10 0 1 2 7 1}


\maketitle

\graphicspath{ {./tmp/} }

\section{Ejercicio realizado en clase.}

\bigskip
A partir del siguiente conjunto de calificaciones académicas, pertenecientes a dos grupos de alumnos (mañana y tarde),
formados por dos notas: teoría y laboratorio, las notas de teoría y laboratorio tendrán valores entre 0 y 5, realizar
un análisis de clasificación no supervisada utilizando el algoritmo \textbf{K-Means}.

\begin{table}[H]
\begin{center}
\begin{tabular}{|c|c|c|c|c|}
\hline
Alumno & Teoría & Laboratorio\\
\hline \hline
A1 & 4 & 4 \\ \hline
A2 & 3 & 5 \\ \hline
A3 & 1 & 2 \\ \hline
A4 & 5 & 5 \\ \hline
A5 & 0 & 1 \\ \hline
A6 & 2 & 2 \\ \hline
A7 & 4 & 4 \\ \hline
A8 & 2 & 1 \\ \hline
\end{tabular}
\end{center}
\end{table}

En primer lugar se introducirán los datos en forma de matriz y se hará la traspuesta de esta.
\begin{Schunk}
\begin{Sinput}
> m<-matrix(c(4,4,3,5,1,2,5,5,0,1,2,2,4,5,2,1),2,8)
> (m<-t(m))
\end{Sinput}
\begin{Soutput}
     [,1] [,2]
[1,]    4    4
[2,]    3    5
[3,]    1    2
[4,]    5    5
[5,]    0    1
[6,]    2    2
[7,]    4    5
[8,]    2    1
\end{Soutput}
\end{Schunk}

\bigskip
En primer lugar se deben seleccionar el número de clusters en los que se van a agrupar los datos,
en este caso serán 2. Además es necesario indicar los centroides iniciales de cada uno de ellos, en
este caso son C1\{0,1\} y C2\{2,2\}. Todo ello es elegido de forma arbitraria.

\bigskip
Introducimos los centroides en una matriz y se realiza la traspuesta.
\begin{Schunk}
\begin{Sinput}
> c<-matrix(c(0,1,2,2),2,2)
> (c<-t(c))
\end{Sinput}
\begin{Soutput}
     [,1] [,2]
[1,]    0    1
[2,]    2    2
\end{Soutput}
\end{Schunk}

\bigskip
La función \texttt{K-Means} se encuentra en el paquete \textbf{stats}. Dicho paquete se carga por defecto al
arrancar R; para comprobarlo se hace uso de la función \texttt{search()}.
\begin{Schunk}
\begin{Sinput}
> search()
\end{Sinput}
\begin{Soutput}
 [1] ".GlobalEnv"        "package:foreign"   "package:stats"    
 [4] "package:graphics"  "package:grDevices" "package:utils"    
 [7] "package:datasets"  "package:methods"   "Autoloads"        
[10] "package:base"     
\end{Soutput}
\end{Schunk}

\bigskip
Por último hacemos uso de la función y obtenemos los centroides finales. Indicamos que el número máximo de
iteraciones es 4.
\begin{Schunk}
\begin{Sinput}
> (clasificacionns<-kmeans(m,c,4))
\end{Sinput}
\begin{Soutput}
K-means clustering with 2 clusters of sizes 4, 4

Cluster means:
  [,1] [,2]
1 1.25 1.50
2 4.00 4.75

Clustering vector:
[1] 2 2 1 2 1 1 2 1

Within cluster sum of squares by cluster:
[1] 3.75 2.75
 (between_SS / total_SS =  84.8 %)

Available components:

[1] "cluster"      "centers"      "totss"        "withinss"     "tot.withinss"
[6] "betweenss"    "size"         "iter"         "ifault"      
\end{Soutput}
\end{Schunk}

\bigskip
Los resultados obtenidos son los mismos que los de clase, es decir, \textbf{C1\{1.25,1.5\} y C2\{4,4.75\}.}

\bigskip
A continuación usaremos los clusters obtenios para separar los datos de la muestra en dos grupos. Para ello
se hace uso de la función \texttt{cbind} la cuál añade (por delante) una columna a la matriz de datos. Dicha
columna se corresponde con la clasificación obtenida, será 1 ó 2 dependiendo del cluster al que pertenezca
cada muestra.
\begin{Schunk}
\begin{Sinput}
> (m = cbind(clasificacionns$cluster,m)) 
\end{Sinput}
\begin{Soutput}
     [,1] [,2] [,3]
[1,]    2    4    4
[2,]    2    3    5
[3,]    1    1    2
[4,]    2    5    5
[5,]    1    0    1
[6,]    1    2    2
[7,]    2    4    5
[8,]    1    2    1
\end{Soutput}
\end{Schunk}

\bigskip
Una vez se tiene el cluster al que pertenece cada muestra, se separa la matriz siguiendo el criterio anterior.
\begin{Schunk}
\begin{Sinput}
> mc1=subset(m,m[,1]==1)
> mc2=subset(m,m[,1]==2)
\end{Sinput}
\end{Schunk}

\bigskip
Por último, limpiamos la columna introducida para el fin buscado y mostramos los dos conjuntos de datos
clusterizados.
\begin{Schunk}
\begin{Sinput}
> (mc1=mc1[,-1])
\end{Sinput}
\begin{Soutput}
     [,1] [,2]
[1,]    1    2
[2,]    0    1
[3,]    2    2
[4,]    2    1
\end{Soutput}
\begin{Sinput}
> (mc2=mc2[,-1])
\end{Sinput}
\begin{Soutput}
     [,1] [,2]
[1,]    4    4
[2,]    3    5
[3,]    5    5
[4,]    4    5
\end{Soutput}
\end{Schunk}

\bigskip
Se puede observar que las muestras 3,5,6,8 pertenecen al mismo grupo, mientras que las muestras 1,2,4,7 
se encuentran en el restante.

\section{Desarrollo por parte del alumno.}

\end{document}
