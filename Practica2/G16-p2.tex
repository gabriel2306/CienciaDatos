\documentclass [a4paper] {article}

\title{R-PL2}
\author{Gabriel L\'opez, Sergio Sanz, \'Alvaro Zamorano}

\usepackage{Sweave}
\begin{document}

\maketitle

\section{Ejercicio realizado en clase.}
Para poder usar el algoritmo \textbf{Apriori} y sus reglas de asociaci\'on vamos a utilizar el paquete
\texttt{arules}. Este paquete hay que descargarlo desde la p\'agina de CRAN y para instalarlo hay que
ejecutar el siguiente c\'odigo:

\begin{Schunk}
\begin{Sinput}
> install.packages("./arules_1.6-4.zip",repos=NULL)
\end{Sinput}
\end{Schunk}

\bigskip
De esta forma, el paquete \'unicamente est\'a instalado. Para poder usarlo es necesario cargarlo:
\begin{Schunk}
\begin{Sinput}
> library(arules)
\end{Sinput}
\end{Schunk}

\end{document}
