\documentclass [a4paper] {article}

%% \usepackage[right=3cm, left=3cm]{geometry}

\usepackage[spanish]{babel} 
\usepackage[utf8]{inputenc} 
\usepackage{multirow} 
\usepackage{float} 

\title{R-PL2}
\author{Gabriel López, Sergio Sanz, Álvaro Zamorano}

\usepackage{Sweave}
\begin{document}

\maketitle

\section{Ejercicio realizado en clase.}
Para poder usar el algoritmo \textbf{Apriori} y sus reglas de asociación vamos a utilizar el paquete
\texttt{arules}. Este paquete hay que descargarlo desde la página de CRAN y para instalarlo hay que
ejecutar el siguiente código:

\begin{Schunk}
\begin{Sinput}
> install.packages("./arules_1.6-4.zip",repos=NULL)
\end{Sinput}
\end{Schunk}

\bigskip
De esta forma, el paquete únicamente está instalado. Para poder usarlo es necesario cargarlo:
\begin{Schunk}
\begin{Sinput}
> library(arules)
\end{Sinput}
\end{Schunk}

Los datos a usar en este primer ejercicio se componen de 6 cestas de la compra, en concreto
estas son: \{Pan, Agua, Leche, Naranjas\},\{{Pan, Agua, Café, Leche\}, \{Pan, Agua, Leche\}, 
\{Pan, Café, Leche\}, \{Pan, Agua\}, \{Leche\}.

\bigskip
Para introducir estos datos en el algoritmo a usar es necesario crear una matriz con el siguiente
aspecto.
\begin{table}[H]
\begin{center}
\begin{tabular}{|c|c|c|c|c|c|}
\hline
Suceso & Pan & Agua & Café & Leche & Naranjas \\
\hline \hline
s1 & 1 & 1 & 0 & 1 & 1 \\ \hline
s2 & 1 & 1 & 1 & 1 & 0 \\ \hline
s3 & 1 & 1 & 0 & 1 & 0 \\ \hline
s4 & 1 & 0 & 1 & 1 & 0 \\ \hline
s5 & 1 & 1 & 0 & 0 & 0 \\ \hline
s6 & 0 & 0 & 0 & 1 & 0 \\ \hline
\end{tabular}
\end{center}
\end{table}

\bigskip
Esta matriz se introduce en R mediante:
\begin{Schunk}
\begin{Sinput}
> muestra<-Matrix(c(1,1,0,1,1,1,1,1,1,0,1,1,0,1,0,1,0,1,1,0,1,1,0,0,0,0,0,0,1,0),
+ 6,5,byrow=T,dimnames=list(c("suceso1","suceso2","suceso3","suceso4","suceso5",
+ "suceso6"),c("Pan","Agua","Cafe","Leche","Naranjas")),sparse=T)
\end{Sinput}
\end{Schunk}

\bigskip
Se necesita convertir la matriz a una matriz dispersa a través de la función \textbf{as} la cuál convierte un objeto a una determinada
clase, en este caso la clase es \textit{nsparseMatrix}. Esta clase lo que hace es cambiar los valores mayores de 0 por un valor binario, 
con el fin de gastar la menor cantidad de memoria posible ya que solo se almacenan aquellas posiciones no vacias, es decir, las que cuyo
valor es distinto de 0.
\begin{Schunk}
\begin{Sinput}
> muestrangCMatrix<-as(muestra,"nsparseMatrix")
\end{Sinput}
\end{Schunk}

\bigskip
El siguiente paso a realizar es calcular la \textbf{traspuesta} de la última matriz generada.
\begin{Schunk}
\begin{Sinput}
> transpuestangCMatrix<-t(muestrangCMatrix)
\end{Sinput}
\end{Schunk}

\bigskip
Antes de aplicar el algoritmo, calculamos y mostramos todas las \textbf{transacciones}, es decir, todas las asociaciones
que hay en nuestros datos.
\begin{Schunk}
\begin{Sinput}
> transacciones<-as(transpuestangCMatrix,"transactions")
> summary(transacciones)
\end{Sinput}
\begin{Soutput}
transactions as itemMatrix in sparse format with
 6 rows (elements/itemsets/transactions) and
 5 columns (items) and a density of 0.5666667 

most frequent items:
     Pan    Leche     Agua     Cafe Naranjas  (Other) 
       5        5        4        2        1        0 

element (itemset/transaction) length distribution:
sizes
1 2 3 4 
1 1 2 2 

   Min. 1st Qu.  Median    Mean 3rd Qu.    Max. 
  1.000   2.250   3.000   2.833   3.750   4.000 

includes extended item information - examples:
  labels
1    Pan
2   Agua
3   Cafe

includes extended transaction information - examples:
  itemsetID
1   suceso1
2   suceso2
3   suceso3
\end{Soutput}
\end{Schunk}

\bigskip
Por último, aplicamos el algoritmo \textbf{Apriori} para las asociaciones cuyo soporte sea igual o superior al 50\% y cuya confianza
sea igual o mayor que el 80\%.
\begin{Schunk}
\begin{Sinput}
> asociaciones<-apriori(transacciones,parameter=list(support=0.5,confidence=0.8))
\end{Sinput}
\end{Schunk}
\begin{Schunk}
\begin{Sinput}
> inspect(asociaciones)
\end{Sinput}
\begin{Soutput}
    lhs             rhs     support   confidence lift count
[1] {}           => {Leche} 0.8333333 0.8333333  1.00 5    
[2] {}           => {Pan}   0.8333333 0.8333333  1.00 5    
[3] {Agua}       => {Pan}   0.6666667 1.0000000  1.20 4    
[4] {Pan}        => {Agua}  0.6666667 0.8000000  1.20 4    
[5] {Leche}      => {Pan}   0.6666667 0.8000000  0.96 4    
[6] {Pan}        => {Leche} 0.6666667 0.8000000  0.96 4    
[7] {Agua,Leche} => {Pan}   0.5000000 1.0000000  1.20 3    
\end{Soutput}
\end{Schunk}

\bigskip
\section{Parte 2.}
\subsection{Datos de ventas de coches.}
Para el leer el fichero \texttt{.txt} hemos creado una función la cuál nos devuelve una lista con las filas
de la matriz.
\begin{Schunk}
\begin{Sinput}
> source("leerMatriz.R")
> leerM
\end{Sinput}
\begin{Soutput}
function(ruta) {

    data<-read.table(ruta,header=TRUE)
    mat<-as.matrix(data)
    sz<-dim(mat)
    l<-c()

    for (i in 1:sz[1]) {
        for (j in 1:sz[2]){
            l<-c(l,mat[i,j])
        }
    }

    return(l)
}
\end{Soutput}
\end{Schunk}

Procedemos a la lectura de dicho fichero.
\begin{Schunk}
\begin{Sinput}
> m<-leerM("2_1.txt")
\end{Sinput}
\end{Schunk}

\bigskip
La matriz leida tiene el siguiente aspecto.
\begin{table}[H]
\begin{center}
\begin{tabular}{|c|c|c|c|c|c|c|}
\hline
Suceso & Xenon & Alarma & Techo & Navegador & Bluetooth & ControlV \\
\hline \hline
s1 & 1 & 0 & 0 & 1 & 1 & 1 \\ \hline
s2 & 1 & 0 & 1 & 0 & 1 & 1 \\ \hline
s3 & 1 & 0 & 0 & 1 & 0 & 1 \\ \hline
s4 & 1 & 0 & 1 & 1 & 1 & 0 \\ \hline
s5 & 1 & 0 & 0 & 0 & 1 & 1 \\ \hline
s6 & 0 & 0 & 0 & 1 & 0 & 0 \\ \hline
s7 & 1 & 0 & 0 & 0 & 1 & 1 \\ \hline
s8 & 0 & 1 & 1 & 0 & 0 & 0 \\ \hline
\end{tabular}
\end{center}
\end{table}

\bigskip
Esta matriz se introduce en R mediante:
\begin{Schunk}
\begin{Sinput}
> mCoches<-Matrix(m,8,6,byrow=T,dimnames=list(c("suceso1","suceso2","suceso3",
+ "suceso4","suceso5","suceso6", "suceso7", "suceso8"),
+ c("Xenon", "Alarma", "Techo", "Navegador", "Bluetooth", "ControlV")),sparse=T)
\end{Sinput}
\end{Schunk}

\bigskip
Se necesita convertir la matriz a una matriz dispersa a través de la función \textbf{as}.
\begin{Schunk}
\begin{Sinput}
> mCochesngC<-as(mCoches,"nsparseMatrix")
\end{Sinput}
\end{Schunk}

\bigskip
El siguiente paso a realizar es calcular la \textbf{traspuesta} de la última matriz generada.
\begin{Schunk}
\begin{Sinput}
> transpuestangC<-t(mCochesngC)
\end{Sinput}
\end{Schunk}

\bigskip
Antes de aplicar el algoritmo, calculamos y mostramos todas las \textbf{transacciones}, es decir, todas las asociaciones
que hay en nuestros datos.
\begin{Schunk}
\begin{Sinput}
> transac<-as(transpuestangC,"transactions")
> summary(transac)
\end{Sinput}
\begin{Soutput}
transactions as itemMatrix in sparse format with
 8 rows (elements/itemsets/transactions) and
 6 columns (items) and a density of 0.5 

most frequent items:
    Xenon Bluetooth  ControlV Navegador     Techo   (Other) 
        6         5         5         4         3         1 

element (itemset/transaction) length distribution:
sizes
1 2 3 4 
1 1 3 3 

   Min. 1st Qu.  Median    Mean 3rd Qu.    Max. 
   1.00    2.75    3.00    3.00    4.00    4.00 

includes extended item information - examples:
  labels
1  Xenon
2 Alarma
3  Techo

includes extended transaction information - examples:
  itemsetID
1   suceso1
2   suceso2
3   suceso3
\end{Soutput}
\end{Schunk}

\bigskip
Por último, aplicamos el algoritmo \textbf{Apriori} para las asociaciones cuyo soporte sea igual o superior al 50\% y cuya confianza
sea igual o mayor que el 80\%.
\begin{Schunk}
\begin{Sinput}
> asoc<-apriori(transac,parameter=list(support=0.4,confidence=0.9))
\end{Sinput}
\end{Schunk}
\begin{Schunk}
\begin{Sinput}
> inspect(asoc)
\end{Sinput}
\begin{Soutput}
    lhs                     rhs     support confidence lift     count
[1] {ControlV}           => {Xenon} 0.625   1          1.333333 5    
[2] {Bluetooth}          => {Xenon} 0.625   1          1.333333 5    
[3] {Bluetooth,ControlV} => {Xenon} 0.500   1          1.333333 4    
\end{Soutput}
\end{Schunk}

\end{document}
