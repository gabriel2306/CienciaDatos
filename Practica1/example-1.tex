\documentclass [a4paper] {article}

\title{R-PL1}
\author{Gabriel L\'opez, Sergio Sanz, \'Alvaro Zamorano}

\usepackage{Sweave}
\begin{document}

\maketitle

En esta parte de la pr\'actica trabajaremos con el fichero
\texttt{satelites.txt}.\\

\bigskip
En primer lugar hay que leer este fichero, para ello usamos
la funci\'on:
\begin{Schunk}
\begin{Sinput}
> satelites<-read.table("satelites.txt")
\end{Sinput}
\end{Schunk}

\bigskip
Para trabajar con la variable radio, y hacer este trabajo m\'as
c\'omodo, la cargamos en una variable:
\begin{Schunk}
\begin{Sinput}
> Radio<-satelites$Radio
\end{Sinput}
\end{Schunk}

\bigskip
En el primer an\'alisis de los datos se cuantifica la \textbf{frecuencia}
de aparici\'on de los mismos. 

%%%%%%%%%%%%%%%%%%%%%%%%%%%%%%%%%%%%%%%%%% save<-capture.output(frabsradio)

\graphicspath{ {C:/Users/tromp/Documents/1.Curso_2019-20/1.Cuatrimetestre/2.Ciencia_Datos/LAB/CienciaDatos/Practica1/} }

\begin{enumerate}
\item
\textit{Frecuencia absoluta: }
\begin{Schunk}
\begin{Sinput}
> frabsradio<-table(Radio)
> save<-toString(frabsradio)  
> source("toPNG.R")
> toPNG(save,"frabsradio.png")