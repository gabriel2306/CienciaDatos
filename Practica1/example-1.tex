\documentclass [a4paper] {article}

\title{R-PL1}
\author{Gabriel L\'opez, Sergio Sanz, \'Alvaro Zamorano}

\usepackage{Sweave}
\begin{document}

\maketitle

En esta parte de la pr\'actica trabajaremos con el fichero
\texttt{satelites.txt}.\\

\bigskip
En primer lugar hay que leer este fichero, para ello usamos
la funci\'on:
\begin{Schunk}
\begin{Sinput}
> satelites<-read.table("satelites.txt")
\end{Sinput}
\end{Schunk}

\bigskip
Para trabajar con la variable radio, y hacer este trabajo m\'as
c\'omodo, la cargamos en una variable:
\begin{Schunk}
\begin{Sinput}
> Radio<-satelites$Radio
\end{Sinput}
\end{Schunk}

\bigskip
En el primer an\'alisis de los datos se cuantifica la \textbf{frecuencia}
de aparici\'on de los mismos. 

\begin{enumerate}
\item
\textit{Frecuencia absoluta: }
\begin{Schunk}
\begin{Sinput}
> frabsradio<-table(Radio)
> frabsradio
\end{Sinput}
\begin{Soutput}
Radio
13 15 16 20 22 27 29 30 33 34 42 
 1  1  1  2  1  1  1  1  1  1  1 
\end{Soutput}
\end{Schunk}

\item
\textit{Frecuencia absoluta acumulada: }
\begin{Schunk}
\begin{Sinput}
> frabsacumradio<-cumsum(table(Radio))
> frabsacumradio
\end{Sinput}
\begin{Soutput}
13 15 16 20 22 27 29 30 33 34 42 
 1  2  3  5  6  7  8  9 10 11 12 
\end{Soutput}
\end{Schunk}

\item
\textit{Frecuencia relativa: }En este caso es necesario crear una funci\'on
para poder calcular este valor. La funci\'on es:
\begin{Schunk}
\begin{Sinput}
> frecrel<-function(Radio){table(Radio)/length(Radio)}
> frecrel(Radio)
\end{Sinput}
\begin{Soutput}
Radio
        13         15         16         20         22         27         29         30         33         34         42 
0.08333333 0.08333333 0.08333333 0.16666667 0.08333333 0.08333333 0.08333333 0.08333333 0.08333333 0.08333333 0.08333333 
\end{Soutput}
\end{Schunk}

\item
\textit{Frecuencia relativa acumulada: }Haremos uso de la funci\'on definida anteriormente:
\begin{Schunk}
\begin{Sinput}
> frecrelacum<-function(Radio){cumsum(table(Radio)/length(Radio))}
> frecrelacum(Radio)
\end{Sinput}
\begin{Soutput}
        13         15         16         20         22         27         29         30         33         34         42 
0.08333333 0.16666667 0.25000000 0.41666667 0.50000000 0.58333333 0.66666667 0.75000000 0.83333333 0.91666667 1.00000000 
\end{Soutput}
\end{Schunk}
\end{enumerate}

\bigskip
El segundo an\'alisis de los datos se basa en calcular la \textbf{media aritm\'etica:}
\begin{Schunk}
\begin{Sinput}
> mr=mean(Radio)
> mr
\end{Sinput}
\begin{Soutput}
[1] 25.08333
\end{Soutput}
\end{Schunk}

\bigskip
El tercer an\'alisis de los datos se basa en calcular las \textbf{medidas de dispersi\'on:}
\begin{enumerate}
\item
\textit{Desviaci\'on t\'ipica: }Para corregir los resultados, se hace el c\'alculo
a trav\'es de:
\begin{Schunk}
\begin{Sinput}
> sdr<-sd(Radio)/sqrt(12/11)
> sdr
\end{Sinput}
\begin{Soutput}
[1] 8.47996
\end{Soutput}
\end{Schunk}

\item
\textit{Varianza: }Al igual que en el caso anterior es necesario corregir el 
resultado por lo que se usa:
\begin{Schunk}
\begin{Sinput}
> varr<-var(Radio)*11/12
> varr
\end{Sinput}
\begin{Soutput}
[1] 71.90972
\end{Soutput}
\end{Schunk}
\end{enumerate}

\bigskip
El cuarto an\'alisis de los datos se basa en las \textbf{medidas de ordenaci\'on,} antes de los c\'alculos es necesario ordenar
los datos en funci\'on de la variable usada, en este caso el radio.
\begin{Schunk}
\begin{Sinput}
> so<-satelites[order(Radio),]
\end{Sinput}
\end{Schunk}

\bigskip
Realmente no ser\'ia necesario ordenar los datos, ya que R se encarga de ello
en caso de no hacerlo. Se procede a realizar los c\'alculos:

\begin{enumerate}
\item
\textit{Mediana:}
\begin{Schunk}
\begin{Sinput}
> mediant<-median(Radio)
> mediant
\end{Sinput}
\begin{Soutput}
[1] 24.5
\end{Soutput}
\end{Schunk}

\item
\textit{Cuartiles:}
\begin{Schunk}
\begin{Sinput}
> cuar1<-quantile(Radio,0.25)
> cuar1
\end{Sinput}
\begin{Soutput}
25% 
 19 
\end{Soutput}
\begin{Sinput}
> cuar2<-quantile(Radio,0.5)
> cuar2
\end{Sinput}
\begin{Soutput}
 50% 
24.5 
\end{Soutput}
\begin{Sinput}
> cuar3<-quantile(Radio,0.75)
> cuar3
\end{Sinput}
\begin{Soutput}
  75% 
30.75 
\end{Soutput}
\begin{Sinput}
> cuar54<-quantile(Radio,0.54)
> cuar54
\end{Sinput}
\begin{Soutput}
 54% 
26.7 
\end{Soutput}
\end{Schunk}
\end{enumerate}

%%%%%%%%%%%%%%%%%%%%%%%%%%%%%%%%%%%%%%%%%%%%%%%%%%%%%%%%%%%%%%%%%%%%%%%%%%%%%%%%%%%%%%%%%%%%%%%%%%%%%%%%%%%%%%%%%%%%%%%%%%%%%%%
\bigskip
A continuaci\'on pasaremos a trabajar con un fichero generado por SPSS, 
\texttt{cardata.sav}.\\

\bigskip
En primer lugar hay que leer este fichero pero no disponemos de la librer\'ia
necesaria para hacerlo, para cargarla usamos:
\begin{Schunk}
\begin{Sinput}
> library(foreign)
\end{Sinput}
\end{Schunk}
Esta librer\'ia se trata de una librer\'ia est\'andar de R.

\bigskip
Una vez cargada procedemos a su lectura
\begin{Schunk}
\begin{Sinput}
> A<-read.spss("cardata.sav")
\end{Sinput}
\end{Schunk}

\bigskip
Para trabajar con la variable mpg, y hacer este trabajo m\'as
c\'omodo, la cargamos en una variable:
\begin{Schunk}
\begin{Sinput}
> mpg<-A$mpg
\end{Sinput}
\end{Schunk}

\bigskip
La variable mpg contiene valores NA, es decir, valores que no se encuentran disponibles 
por lo que es imposible realizar c\'alculos con ella. Para eliminar estos valores usamos:
\begin{Schunk}
\begin{Sinput}
> mpg<-mpg[!is.na(mpg)]
\end{Sinput}
\end{Schunk}

\bigskip
En el primer an\'alisis de los datos se cuantifica la \textbf{frecuencia}
de aparici\'on de los mismos. 

\begin{enumerate}
\item
\textit{Frecuencia absoluta: }
\begin{Schunk}
\begin{Sinput}
> frabsmpg<-table(mpg)
> frabsmpg
\end{Sinput}
\begin{Soutput}
mpg
15.5 16.2 16.5 16.9   17 17.5 17.6 17.7 18.1 18.2 18.5 18.6 19.1 19.2 19.4 19.8 19.9 20.2 20.3 20.5 20.6 20.8 21.1 21.5 21.6   22 22.3 22.4   23 23.2 23.5 23.6 23.7 
   1    1    1    1    2    1    2    1    2    1    1    1    1    3    2    1    1    4    1    2    2    1    1    1    1    1    1    1    2    1    1    1    1 
23.8 23.9   24 24.2 24.3   25 25.1 25.4 25.8   26 26.4 26.6 26.8   27 27.2 27.4 27.5 27.9   28 28.1 28.4 28.8   29 29.5 29.8 29.9   30 30.4 30.7 30.9   31 31.3 31.5 
   1    2    1    1    1    1    1    2    1    1    1    2    1    4    3    1    1    1    3    1    1    1    1    1    2    1    2    1    1    1    3    1    1 
31.6 31.8 31.9   32 32.1 32.2 32.3 32.4 32.7 32.8 32.9   33 33.5 33.7 33.8   34 34.1 34.2 34.3 34.4 34.5 34.7   35 35.1 35.7   36 36.1 36.4   37 37.2 37.3 37.7   38 
   1    1    1    3    1    1    1    2    1    1    1    1    1    1    1    2    2    1    1    1    2    1    1    1    1    5    2    1    3    1    1    1    4 
38.1   39 39.1 39.4 40.8 40.9 41.5 43.1 43.4   44 44.3 44.6 46.6 
   1    1    1    1    1    1    1    1    1    1    1    1    1 
\end{Soutput}
\end{Schunk}

\item
\textit{Frecuencia absoluta acumulada: }
\begin{Schunk}
\begin{Sinput}
> frabsacummpg<-cumsum(table(mpg))
> frabsacummpg
\end{Sinput}
\begin{Soutput}
15.5 16.2 16.5 16.9   17 17.5 17.6 17.7 18.1 18.2 18.5 18.6 19.1 19.2 19.4 19.8 19.9 20.2 20.3 20.5 20.6 20.8 21.1 21.5 21.6   22 22.3 22.4   23 23.2 23.5 23.6 23.7 
   1    2    3    4    6    7    9   10   12   13   14   15   16   19   21   22   23   27   28   30   32   33   34   35   36   37   38   39   41   42   43   44   45 
23.8 23.9   24 24.2 24.3   25 25.1 25.4 25.8   26 26.4 26.6 26.8   27 27.2 27.4 27.5 27.9   28 28.1 28.4 28.8   29 29.5 29.8 29.9   30 30.4 30.7 30.9   31 31.3 31.5 
  46   48   49   50   51   52   53   55   56   57   58   60   61   65   68   69   70   71   74   75   76   77   78   79   81   82   84   85   86   87   90   91   92 
31.6 31.8 31.9   32 32.1 32.2 32.3 32.4 32.7 32.8 32.9   33 33.5 33.7 33.8   34 34.1 34.2 34.3 34.4 34.5 34.7   35 35.1 35.7   36 36.1 36.4   37 37.2 37.3 37.7   38 
  93   94   95   98   99  100  101  103  104  105  106  107  108  109  110  112  114  115  116  117  119  120  121  122  123  128  130  131  134  135  136  137  141 
38.1   39 39.1 39.4 40.8 40.9 41.5 43.1 43.4   44 44.3 44.6 46.6 
 142  143  144  145  146  147  148  149  150  151  152  153  154 
\end{Soutput}
\end{Schunk}

\item
\textit{Frecuencia relativa: }En este caso es necesario crear una funci\'on
para poder calcular este valor. La funci\'on es:
\begin{Schunk}
\begin{Sinput}
> frecrel<-function(mpg){table(mpg)/length(mpg)}
> frecrel(mpg)
\end{Sinput}
\begin{Soutput}
mpg
       15.5        16.2        16.5        16.9          17        17.5        17.6        17.7        18.1        18.2        18.5        18.6        19.1 
0.006493506 0.006493506 0.006493506 0.006493506 0.012987013 0.006493506 0.012987013 0.006493506 0.012987013 0.006493506 0.006493506 0.006493506 0.006493506 
       19.2        19.4        19.8        19.9        20.2        20.3        20.5        20.6        20.8        21.1        21.5        21.6          22 
0.019480519 0.012987013 0.006493506 0.006493506 0.025974026 0.006493506 0.012987013 0.012987013 0.006493506 0.006493506 0.006493506 0.006493506 0.006493506 
       22.3        22.4          23        23.2        23.5        23.6        23.7        23.8        23.9          24        24.2        24.3          25 
0.006493506 0.006493506 0.012987013 0.006493506 0.006493506 0.006493506 0.006493506 0.006493506 0.012987013 0.006493506 0.006493506 0.006493506 0.006493506 
       25.1        25.4        25.8          26        26.4        26.6        26.8          27        27.2        27.4        27.5        27.9          28 
0.006493506 0.012987013 0.006493506 0.006493506 0.006493506 0.012987013 0.006493506 0.025974026 0.019480519 0.006493506 0.006493506 0.006493506 0.019480519 
       28.1        28.4        28.8          29        29.5        29.8        29.9          30        30.4        30.7        30.9          31        31.3 
0.006493506 0.006493506 0.006493506 0.006493506 0.006493506 0.012987013 0.006493506 0.012987013 0.006493506 0.006493506 0.006493506 0.019480519 0.006493506 
       31.5        31.6        31.8        31.9          32        32.1        32.2        32.3        32.4        32.7        32.8        32.9          33 
0.006493506 0.006493506 0.006493506 0.006493506 0.019480519 0.006493506 0.006493506 0.006493506 0.012987013 0.006493506 0.006493506 0.006493506 0.006493506 
       33.5        33.7        33.8          34        34.1        34.2        34.3        34.4        34.5        34.7          35        35.1        35.7 
0.006493506 0.006493506 0.006493506 0.012987013 0.012987013 0.006493506 0.006493506 0.006493506 0.012987013 0.006493506 0.006493506 0.006493506 0.006493506 
         36        36.1        36.4          37        37.2        37.3        37.7          38        38.1          39        39.1        39.4        40.8 
0.032467532 0.012987013 0.006493506 0.019480519 0.006493506 0.006493506 0.006493506 0.025974026 0.006493506 0.006493506 0.006493506 0.006493506 0.006493506 
       40.9        41.5        43.1        43.4          44        44.3        44.6        46.6 
0.006493506 0.006493506 0.006493506 0.006493506 0.006493506 0.006493506 0.006493506 0.006493506 
\end{Soutput}
\end{Schunk}

\item
\textit{Frecuencia relativa acumulada: }Haremos uso de la funci\'on definida anteriormente:
\begin{Schunk}
\begin{Sinput}
> frecrelacum<-function(mpg){cumsum(table(mpg)/length(mpg))}
> frecrelacum(mpg)
\end{Sinput}
\begin{Soutput}
       15.5        16.2        16.5        16.9          17        17.5        17.6        17.7        18.1        18.2        18.5        18.6        19.1 
0.006493506 0.012987013 0.019480519 0.025974026 0.038961039 0.045454545 0.058441558 0.064935065 0.077922078 0.084415584 0.090909091 0.097402597 0.103896104 
       19.2        19.4        19.8        19.9        20.2        20.3        20.5        20.6        20.8        21.1        21.5        21.6          22 
0.123376623 0.136363636 0.142857143 0.149350649 0.175324675 0.181818182 0.194805195 0.207792208 0.214285714 0.220779221 0.227272727 0.233766234 0.240259740 
       22.3        22.4          23        23.2        23.5        23.6        23.7        23.8        23.9          24        24.2        24.3          25 
0.246753247 0.253246753 0.266233766 0.272727273 0.279220779 0.285714286 0.292207792 0.298701299 0.311688312 0.318181818 0.324675325 0.331168831 0.337662338 
       25.1        25.4        25.8          26        26.4        26.6        26.8          27        27.2        27.4        27.5        27.9          28 
0.344155844 0.357142857 0.363636364 0.370129870 0.376623377 0.389610390 0.396103896 0.422077922 0.441558442 0.448051948 0.454545455 0.461038961 0.480519481 
       28.1        28.4        28.8          29        29.5        29.8        29.9          30        30.4        30.7        30.9          31        31.3 
0.487012987 0.493506494 0.500000000 0.506493506 0.512987013 0.525974026 0.532467532 0.545454545 0.551948052 0.558441558 0.564935065 0.584415584 0.590909091 
       31.5        31.6        31.8        31.9          32        32.1        32.2        32.3        32.4        32.7        32.8        32.9          33 
0.597402597 0.603896104 0.610389610 0.616883117 0.636363636 0.642857143 0.649350649 0.655844156 0.668831169 0.675324675 0.681818182 0.688311688 0.694805195 
       33.5        33.7        33.8          34        34.1        34.2        34.3        34.4        34.5        34.7          35        35.1        35.7 
0.701298701 0.707792208 0.714285714 0.727272727 0.740259740 0.746753247 0.753246753 0.759740260 0.772727273 0.779220779 0.785714286 0.792207792 0.798701299 
         36        36.1        36.4          37        37.2        37.3        37.7          38        38.1          39        39.1        39.4        40.8 
0.831168831 0.844155844 0.850649351 0.870129870 0.876623377 0.883116883 0.889610390 0.915584416 0.922077922 0.928571429 0.935064935 0.941558442 0.948051948 
       40.9        41.5        43.1        43.4          44        44.3        44.6        46.6 
0.954545455 0.961038961 0.967532468 0.974025974 0.980519481 0.987012987 0.993506494 1.000000000 
\end{Soutput}
\end{Schunk}
\end{enumerate}

\bigskip
El segundo an\'alisis de los datos se basa en calcular la \textbf{media aritm\'etica:}
\begin{Schunk}
\begin{Sinput}
> mm<-mean(mpg)
> mm
\end{Sinput}
\begin{Soutput}
[1] 28.79351
\end{Soutput}
\end{Schunk}

\bigskip
El tercer an\'alisis de los datos se basa en calcular las \textbf{medidas de dispersi\'on:}
\begin{enumerate}
\item
\textit{Desviaci\'on t\'ipica: }Para corregir los resultados, se hace el c\'alculo
a trav\'es de:
\begin{Schunk}
\begin{Sinput}
> sdm<-sd(mpg)/sqrt(12/11)
> sdm
\end{Sinput}
\begin{Soutput}
[1] 7.063141
\end{Soutput}
\end{Schunk}

\item
\textit{Varianza: }Al igual que en el caso anterior es necesario corregir el 
resultado por lo que se usa:
\begin{Schunk}
\begin{Sinput}
> varm<-var(mpg)*11/12
> varm
\end{Sinput}
\begin{Soutput}
[1] 49.88796
\end{Soutput}
\end{Schunk}
\end{enumerate}

\bigskip
El cuarto an\'alisis de los datos se basa en las \textbf{medidas de ordenaci\'on:,}
\bigskip

\begin{enumerate}
\item
\textit{Mediana:}
\begin{Schunk}
\begin{Sinput}
> mediantm<-median(mpg)
> mediantm
\end{Sinput}
\begin{Soutput}
[1] 28.9
\end{Soutput}
\end{Schunk}

\item
\textit{Cuartiles:}
\begin{Schunk}
\begin{Sinput}
> cuar1m<-quantile(mpg,0.25)
> cuar1m
\end{Sinput}
\begin{Soutput}
  25% 
22.55 
\end{Soutput}
\begin{Sinput}
> cuar2m<-quantile(mpg,0.5)
> cuar2m
\end{Sinput}
\begin{Soutput}
 50% 
28.9 
\end{Soutput}
\begin{Sinput}
> cuar3m<-quantile(mpg,0.75)
> cuar3m
\end{Sinput}
\begin{Soutput}
   75% 
34.275 
\end{Soutput}
\begin{Sinput}
> cuar54m<-quantile(mpg,0.54)
> cuar54m
\end{Sinput}
\begin{Soutput}
54% 
 30 
\end{Soutput}
\end{Schunk}
\end{enumerate}

\end{document}
