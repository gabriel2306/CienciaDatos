\documentclass [a4paper] {article}

\title{R-PL1}
\author{Gabriel L\'opez, Sergio Sanz, \'Alvaro Zamorano}

\usepackage{Sweave}
\begin{document}

\maketitle

En esta parte de la pr\'actica trabajaremos con el fichero
\texttt{satelites.txt}.\\

\bigskip
En primer lugar hay que leer este fichero, para ello usamos
la funci\'on:
\begin{Schunk}
\begin{Sinput}
> satelites<-read.table("satelites.txt")
\end{Sinput}
\end{Schunk}

\bigskip
Para trabajar con la variable radio, y hacer este trabajo m\'as
c\'omodo, la cargamos en una variable:
\begin{Schunk}
\begin{Sinput}
> Radio<-satelites$Radio
\end{Sinput}
\end{Schunk}

\bigskip
En el primer an\'alisis de los datos se cuantifica la \textbf{frecuencia}
de aparici\'on de los mismos. 

\begin{enumerate}
\item
\textit{Frecuencia absoluta: }
\begin{Schunk}
\begin{Sinput}
> frabsradio<-table(Radio)
\end{Sinput}
\end{Schunk}

\item
\textit{Frecuencia absoluta acumulada: }
\begin{Schunk}
\begin{Sinput}
> frabsacumradio<-cumsum(table(Radio))
\end{Sinput}
\end{Schunk}

\item
\textit{Frecuencia relativa: }En este caso es necesario crear una funci\'on
para poder calcular este valor. La funci\'on es:
\begin{Schunk}
\begin{Sinput}
> frecrel<-function(Radio){table(Radio)/length(Radio)}
\end{Sinput}
\end{Schunk}

\item
\textit{Frecuencia relativa acumulada: }Haremos uso de la funci\'on definida anteriormente:
\begin{Schunk}
\begin{Sinput}
> frecrelacum<-function(Radio){cumsum(table(Radio)/length(Radio))}
\end{Sinput}
\end{Schunk}
\end{enumerate}

\bigskip
El segundo an\'alisis de los datos se basa en calcular la \textbf{media aritm\'etica:}
\begin{Schunk}
\begin{Sinput}
> mr=mean(Radio)
\end{Sinput}
\end{Schunk}

\bigskip
El tercer an\'alisis de los datos se basa en calcular las \textbf{medidas de dispersi\'on:}
\begin{enumerate}
\item
\textit{Desviaci\'on t\'ipica: }Para corregir los resultados, se hace el c\'alculo
a trav\'es de:
\begin{Schunk}
\begin{Sinput}
> sdr=sd(Radio)/sqrt(12/11)
\end{Sinput}
\end{Schunk}

\item
\textit{Varianza: }Al igual que en el caso anterior es necesario corregir el 
resultado por lo que se usa:
\begin{Schunk}
\begin{Sinput}
> varr=var(Radio)*11/12
\end{Sinput}
\end{Schunk}
\end{enumerate}

\bigskip
El cuarto an\'alisis de los datos se basa en las \textbf{medidas de ordenaci\'on:}, antes de los c\'alculos es necesario ordenar
los datos en funci\'on de la variable usada, en este caso el radio.
\begin{Schunk}
\begin{Sinput}
> so=s[order(s$Radio),]
\end{Sinput}
\end{Schunk}

\bigskip
Una vez ordenados los datos se puede proceder a calcular:

\begin{enumerate}
\item
\textit{Mediana:}
\begin{Schunk}
\begin{Sinput}
> mediant=median(s$Radio)
\end{Sinput}
\end{Schunk}

\item
\textit{Cuartiles:}
\begin{Schunk}
\begin{Sinput}
> cuar1=quantile(s$Radio,0.25)
> cuar2=quantile(s$Radio,0.5)
> cuar3=quantile(s$Radio,0.75)
> cuar54=quantile(s$Radio,0.54)
\end{Sinput}
\end{Schunk}
\end{enumerate}

\end{document}
